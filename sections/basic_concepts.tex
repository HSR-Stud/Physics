% !TeX spellcheck = en_US
\section{Basic concepts}

\subsection{Models of atoms}
The best model of an atom is the \emph{Schroedinger Equation}
\begin{equation}
	i \hbar \frac{\partial\Psi(x,t)}{\partial t} = U(x) \cdot \Psi(x,t) - \frac{\hbar}{2m} \frac{\partial^2 \Psi(x,t)}{\partial x^2}
\end{equation}


An easier model is \emph{Bohr's model}

\begin{figure}[ht!]
    \begin{subfigure}[c]{0.48\linewidth}
        \centering
        \begin{tikzpicture}

    % Nucleus
    \node[draw=HSRBlue,fill=HSRBlue80,circle,inner sep=2pt] (nucleus) at (0,0) {\tiny\textcolor{HSRWhite}{$\mathbf{+}$}};
    \node[draw=HSRBlue,circle,minimum size=1cm] (largenuc) at (3.2,1) {};
    \draw[<-] (nucleus) -- (largenuc);
    \node[below of=largenuc] {Nucleus};
    
    \node[draw=HSRBlue,,circle,inner sep=0pt] at (3.4,1) {\tiny\textcolor{HSRBlue}{+}};
    \node[draw=HSRBlue,,circle,inner sep=0pt] at (3.1,1.2) {\tiny\textcolor{HSRBlue}{+}};
    \node[draw=HSRBlue,,circle,inner sep=0pt] at (3.0,0.8) {\tiny\textcolor{HSRBlue}{+}};
    \node[draw=black,fill=black,circle,inner sep=2pt] at (3.4,1.3) {};
    \node[draw=black,fill=black,circle,inner sep=2pt] at (3.3,0.7) {};
    \node[draw=black,fill=black,circle,inner sep=2pt] at (2.85,1.1) {};

    
    % K-Shell
    \draw[very thick,draw] (0,0) circle (1.0);
    \node at (-0.6,-0.5) {\Large K};
    \node at (0.6,-0.4) {1s};
    \pgfmathparse{1/sqrt(2)}
    \draw[draw=HSRBlue,fill=HSRBlue] (\pgfmathresult,-\pgfmathresult) circle (0.1);
    \draw[draw=HSRBlue,fill=HSRBlue] (-\pgfmathresult,\pgfmathresult) circle (0.1);
    
    % L-Shell
    \draw[very thick,draw] (0,0) circle (1.6);
    \draw[very thick,draw] (0,0) circle (1.8);
    \node at (-1.7,-1) {\Large L};
    \node at (1.3,-0.4) {2s};
    \node at (1.8,-1) {2p};
    
    \draw[draw=HSRBlue,fill=HSRBlue] (0,1.6) circle (0.1);
    \draw[draw=HSRBlue,fill=HSRBlue] (0,-1.6) circle (0.1);
    \pgfmathparse{1.8/sqrt(2)}
    \draw[draw=HSRBlue,fill=HSRBlue] (\pgfmathresult,\pgfmathresult) circle (0.1);
    \draw[draw=HSRBlue,fill=HSRBlue] (-\pgfmathresult,-\pgfmathresult) circle (0.1);


\end{tikzpicture}
        \caption{Example: Helium}
    \end{subfigure}
    \begin{subfigure}[c]{0.48\linewidth}
        \centering
        \begin{tabular}{llllll}
            \toprule
            & & \multicolumn{4}{c}{Subshell} \\
            & \multicolumn{1}{r}{l=} & 0 & 1 & 2 & 3 \\ \cmidrule{2-6}
            n & Shell & s & p & d & f \\
            \midrule
            1 & K & 2 & & & \\
            2 & L & 2 & 6 & & \\
            3 & M & 2 & 6 & 10 & \\
            4 & N & 2 & 6 & 10 & 14 \\
            \bottomrule
        \end{tabular}
        \caption{Max. possible number of electrons in shells and subshells}
    \end{subfigure}
    \caption{Bohr's model}
\end{figure}

$n$ is the principal quantum number and $l$ is the orbital angular momentum.
The angular momentum of an orbiting electron is calculated by
\begin{equation}
	L_n = r m_e v = n \frac{h}{2\pi} = n \hbar \qquad n = 1,2,3,\dots
\end{equation}

%TODO: weitere gleichungen ergänzen?

The \emph{Bohr radius} specifies the distance between the nucleus and the first shell in a hydrogen atom.
\begin{equation}
	r_0 = a_0 = \frac{\varepsilon_0 h^2}{\pi m_e q_e^2}
\end{equation}

\subsection{Mean kinetic energy and temperature}
Ideal gas equation
\begin{equation}
	PV = \frac{1}{3} Nm \cdot \bar{v^2} = \frac{N}{N_A}RT = N k_B T
\end{equation}

where $P$ is the gas pressure, $V$ the gas volume, $N$ the number of molecules in the gas,
$m$ the mass of a gas molecule, $\bar{v^2}$ the mean square velocity, $\rho$ the density of the gas.

Average kinetic energy per molecule
\begin{equation}
	E_{kin} = \frac{1}{2} m \bar{v^2} = \frac{3}{2} k_B T
\end{equation}

Gas pressure in the kinetic theory
\begin{equation}
	P = \frac{1}{3} \rho \bar{v^2}
\end{equation}

Root mean square velocity
\begin{equation}
	v_{rms} = \sqrt{\frac{3 k_B T}{m}} = \sqrt{\frac{3RT}{M}}
\end{equation}

\subsection{Electric Noise}
Root mean square voltage
\begin{equation}
	v_{rms}  = \sqrt{4 k_B T R B}
\end{equation}
where $B$ is the Bandwith of the $RC$ network.

\subsection{Molecular velocity and energy distribution}
Velocity distribution function
\begin{equation}
	n_v = N \cdot 4 \pi \left(\frac{m}{2\pi k_B T}\right)^2 v^2 e^{-\left(\frac{m v^2}{2 k_B T}\right)}
\end{equation}

Boltzmann factor
\begin{align}
	e^{-\left(\frac{E}{k_B T}\right)} \\
	\frac{n_E}{N} = C_E \cdot e^{-\left(\frac{E}{k_B T}\right)}
\end{align}

\subsection{Energy of a photon}

\begin{equation}
	E = h \cdot f = \hbar \cdot \omega = \frac{h \cdot c}{\lambda}
\end{equation}

Where the wave length $\lambda = \frac{c}{f}$. 