% !TeX spellcheck = en_US
\section{Basic concepts}

\subsection{Models of atoms\buch{4}}
The best model of an atom is the \emph{Schroedinger Equation}
\begin{equation}
	i \hbar \frac{\partial\Psi(x,t)}{\partial t} = U(x) \cdot \Psi(x,t) - \frac{\hbar}{2m} \frac{\partial^2 \Psi(x,t)}{\partial x^2}
\end{equation}


An easier model is \emph{Bohr's model}

\begin{figure}[ht!]
    \begin{subfigure}[c]{0.48\linewidth}
        \centering
        \begin{tikzpicture}

    % Nucleus
    \node[draw=HSRBlue,fill=HSRBlue80,circle,inner sep=2pt] (nucleus) at (0,0) {\tiny\textcolor{HSRWhite}{$\mathbf{+}$}};
    \node[draw=HSRBlue,circle,minimum size=1cm] (largenuc) at (3.2,1) {};
    \draw[<-] (nucleus) -- (largenuc);
    \node[below of=largenuc] {Nucleus};
    
    \node[draw=HSRBlue,,circle,inner sep=0pt] at (3.4,1) {\tiny\textcolor{HSRBlue}{+}};
    \node[draw=HSRBlue,,circle,inner sep=0pt] at (3.1,1.2) {\tiny\textcolor{HSRBlue}{+}};
    \node[draw=HSRBlue,,circle,inner sep=0pt] at (3.0,0.8) {\tiny\textcolor{HSRBlue}{+}};
    \node[draw=black,fill=black,circle,inner sep=2pt] at (3.4,1.3) {};
    \node[draw=black,fill=black,circle,inner sep=2pt] at (3.3,0.7) {};
    \node[draw=black,fill=black,circle,inner sep=2pt] at (2.85,1.1) {};

    
    % K-Shell
    \draw[very thick,draw] (0,0) circle (1.0);
    \node at (-0.6,-0.5) {\Large K};
    \node at (0.6,-0.4) {1s};
    \pgfmathparse{1/sqrt(2)}
    \draw[draw=HSRBlue,fill=HSRBlue] (\pgfmathresult,-\pgfmathresult) circle (0.1);
    \draw[draw=HSRBlue,fill=HSRBlue] (-\pgfmathresult,\pgfmathresult) circle (0.1);
    
    % L-Shell
    \draw[very thick,draw] (0,0) circle (1.6);
    \draw[very thick,draw] (0,0) circle (1.8);
    \node at (-1.7,-1) {\Large L};
    \node at (1.3,-0.4) {2s};
    \node at (1.8,-1) {2p};
    
    \draw[draw=HSRBlue,fill=HSRBlue] (0,1.6) circle (0.1);
    \draw[draw=HSRBlue,fill=HSRBlue] (0,-1.6) circle (0.1);
    \pgfmathparse{1.8/sqrt(2)}
    \draw[draw=HSRBlue,fill=HSRBlue] (\pgfmathresult,\pgfmathresult) circle (0.1);
    \draw[draw=HSRBlue,fill=HSRBlue] (-\pgfmathresult,-\pgfmathresult) circle (0.1);


\end{tikzpicture}
        \caption{Example: Helium}
    \end{subfigure}
    \begin{subfigure}[c]{0.48\linewidth}
        \centering
        \begin{tabular}{llllll}
            \toprule
            & & \multicolumn{4}{c}{Subshell} \\
            & \multicolumn{1}{r}{l=} & 0 & 1 & 2 & 3 \\ \cmidrule{2-6}
            n & Shell & s & p & d & f \\
            \midrule
            1 & K & 2 & & & \\
            2 & L & 2 & 6 & & \\
            3 & M & 2 & 6 & 10 & \\
            4 & N & 2 & 6 & 10 & 14 \\
            \bottomrule
        \end{tabular}
        \caption{Max. possible number of electrons in shells and subshells}
    \end{subfigure}
    \caption{Bohr's model}
\end{figure}

$n$ is the principal quantum number and $l$ is the orbital angular momentum.
The angular momentum of an orbiting electron is calculated by
\begin{equation}
	L_n = r m_e v = n \frac{h}{2\pi} = n \hbar \qquad n = 1,2,3,\dots
\end{equation}

%TODO: weitere gleichungen ergänzen?

The \emph{Bohr radius} specifies the distance between the nucleus and the first shell in a hydrogen atom.
\begin{equation}
	r_0 = a_0 = \frac{\varepsilon_0 h^2}{\pi m_e q_e^2}
\end{equation}

\subsection{Mean kinetic energy and temperature\buch{25}}
Ideal gas equation
\begin{equation}
	PV = \frac{1}{3} Nm \cdot \bar{v^2} = \frac{N}{N_A}RT = N k_B T
\end{equation}

where $P$ is the gas pressure, $V$ the gas volume, $N$ the number of molecules in the gas,
$m$ the mass of a gas molecule, $\bar{v^2}$ the mean square velocity, $\rho$ the density of the gas.

Average kinetic energy per molecule
\begin{equation}
	E_{kin} = \frac{1}{2} m \bar{v^2} = \frac{3}{2} k_B T
\end{equation}

Gas pressure in the kinetic theory
\begin{equation}
	P = \frac{1}{3} \rho \bar{v^2}
\end{equation}

Root mean square velocity
\begin{equation}
	v_{rms} = \sqrt{\frac{3 k_B T}{m}} = \sqrt{\frac{3RT}{M}}
\end{equation}

Maxwell's priciple of equipartition of energy assigns an average of $\frac{1}{2}k_b T$ to 
each degree of freedom (e.g. translation in x,y,z, rotation in x,y,z, \ldots).
The Rule of Dulong-Petit for the molar heat capacity of a solid is $C_m = 3R$.


\subsection{Electric Noise\buch{44}}
Root mean square voltage
\begin{equation}
	v_{rms}  = \sqrt{4 k_B T R B}
\end{equation}
where $B$ is the Bandwith of the $RC$ network.

\subsection{Molecular velocity and energy distribution\buch{40}}
Velocity distribution function
\begin{equation}
	n_v(v) = N \cdot 4 \pi \left(\frac{m}{2\pi k_B T}\right)^2 v^2 e^{-\left(\frac{m v^2}{2 k_B T}\right)}
\end{equation}

Energy distribution function
\begin{equation}
    n_E(E) = N \frac{2}{\sqrt{\pi}} \left(\frac{1}{k_B T}\right)^{3/2} E^{1/2} e^{-\frac{E}{k_B T}}
\end{equation}

Boltzmann factor
\begin{align}
	e^{-\left(\frac{E}{k_B T}\right)} \\
	\frac{n_E}{N} = C_E \cdot e^{-\left(\frac{E}{k_B T}\right)}
\end{align}

\subsection{Thermally activated process\buch{45}}
Many physical and chemical processes exhibit \emph{Arrhenius} type behavior, i.e. 
the rate of change is proportional to $\exp(-E_A / k_B T)$ where $E_A$ is the activation energy.

\subsubsection{Atomic diffusion}

\begin{figure}[ht]
    \centering
    \begin{tikzpicture}[scale=0.5,transform shape,
    atom/.style = {draw=HSRBlue,fill=HSRBlue60,circle,minimum height=1cm},
    impurity/.style = {draw=black,fill=black,circle,minimum height=0.45cm}
]

\begin{scope}[shift={(-6,0)}]
    \node at (3,5) {\Large\textcolor{black}{$A$}};
    \foreach \x in {1,...,5} {
        \foreach \y in {1,...,4} {
            \node[atom] at (\x,\y) {};
        }
    }
    \node[impurity] at (2.5,2.5) {};
\end{scope}

\begin{scope}
    \node at (3,5) {\Large\textcolor{black}{$A^*$}};
    \foreach \x in {1,...,5} {
        \foreach \y in {1,...,4} {
            \ifthenelse{\NOT\x=3 \OR \y=1 \OR \y=4}{
                \node[atom] at (\x,\y) {};
            }
            {
                \ifthenelse{\y=3}{\def\offset{0.1}}{\def\offset{-0.1}}
                \node[draw=HSRBlue,fill=HSRBlue,ellipse,minimum width=1cm, minimum height=0.8cm] at (\x,{\y+\offset}) {};
            }
        }
    }
    \node[impurity] at (3.0,2.5) {};
\end{scope}

\begin{scope}[shift={(6,0)}]
    \node at (3,5) {\Large\textcolor{black}{$B$}};
    \foreach \x in {1,...,5} {
        \foreach \y in {1,...,4} {
            \node[atom] at (\x,\y) {};
        }
    }
    \node[impurity] at (3.5,2.5) {};
\end{scope}

\end{tikzpicture}
    \caption{Atomic diffusion of impurity atoms}
\end{figure}

In diffusion of impurity atoms, the activation energy is $E_A = U_{A^*} - U_A$ where 
$U_A$ is the potential at the original point $A$ and $U_{A^*}$ is the potential at
the barrier between two sites. This leads to the \emph{frequency of jumps} $\vartheta$
\begin{equation}
    \vartheta = \text{const} \cdot v_0 \exp\left(-\frac{E_A}{k_B}\right)
\end{equation}

The total square displacement $L$ of an impurity atom after $N$ jumps is
\begin{equation}
    L^2 = X^2 + Y^2 = a^2 \vartheta t = 2 D t
\end{equation}
where $D$ is the diffusion coefficient
\begin{equation}
    D = \frac{1}{2} a^2 \vartheta = \frac{1}{2} a^2 v_0 C \exp\left(-\frac{E_A}{k_B T}\right) = D_0 \exp\left(-\frac{E_A}{k_B T}\right)
\end{equation}


\subsection{Energy of a photon}

\begin{equation}
	E = h \cdot f = \hbar \cdot \omega = \frac{h \cdot c}{\lambda}
\end{equation}
where the wave length $\lambda = \frac{c}{f}$. 

%TODO add diagram from slide 22
Adsorption of a photon lifts an electron from an initial energy state $E_i$ to a higher final
energy state $E_f = E_i + hf$ where $hf$ is the photon energy.
A photon with energy $hf$ is emitted when an electron drops from a higher initial state
$E_i$ to a lower state $E_f$

\subsubsection{Energy levels of hydrogen}

\begin{equation}
    E_n = -\frac{h c R_y}{n^2}
\end{equation}
where $n = 1,2,3,\ldots$ and $R_y$ is the Rydberg constant $R_y = \frac{m_e q_e^4}{\varepsilon_0^2 8 h^3 c}$=\SI{1.097e7}{\meter\tothe{-1}}

\subsection{From atomic levels to energy bands}
The electrons in an atom populate \emph{discrete} energy levels.
When atoms for a crystal lattice, many electrons close together would occupy
identical energy levels, which is not possible (exclusion principle of Pauli).
The individual energy levels will therefore vary slightly, creating an \emph{energy band}
%TODO add diagram from slide 25

