% !TeX spellcheck = en_US
\section{Piezoelectricity}
\subsection{Basics of piezoelectricity\buch{638}}
Certain crystals (e.g. quartz $SiO_2$, $BaTiO_3$, \ldots) become polarized
when they are mechanically stressed and exhibit mechanical strain when
they experience an electrical field.
This requires a special crystal structure, i.e. \emph{no center of symmetry}.

The induced polarization $P$ along $i$ by an applied stress $T$ along $j$ is
\begin{equation}
    P_i = d_{ij} T_j
\end{equation}

The induced strain $S$ along $j$ by an applied electric field $E$ along $i$ is
\begin{equation}
    S_j = d_{ij} E_i
\end{equation}

%TODO: Add image - centro-symmetric / noncentro-symmetric

The efficiency of the conversion between mechanical and electrical energy is
expressed as the \emph{electromechanical coupling factor} k
\begin{equation}
    k^2 = \frac{\text{Electrical energy conv. to mechanical energy}}{\text{Input of electrical energy}} = \frac{\text{Mechanical energy conv. to electrical energy}}{\text{Input of mechanical energy}}
\end{equation}

The $k$ and $d$ values of some piezo materials are depicted in appendix~\ref{app:piezo}.

\subsection{Applications of piezoelectricity\buch{644}}
\paragraph{Quartz oscillator}
%TODO: draw image
The mechanical resonant frequency is
\begin{equation}
    f_s = \frac{1}{2\pi\sqrt{LC}} = \frac{1}{2\pi\sqrt{\frac{m}{k}}}
\end{equation}
The electrodes lead to an antiresonant frequency
\begin{equation}
    f_a = \frac{1}{2\pi\sqrt{L\frac{C_0C}{C_0+C}}}
\end{equation}

\paragraph{Transducer}
%TODO: add image

\paragraph{Actuator}
%TODO: add image

\subsection{Ferroelectricity\buch{647}}
Certain crystals are permanently polarized. 
These crystals are called ferroelectric.
The critical temperature above which ferroelectric property is lost is called
the \emph{Curie temperature} ($T_C$).

\subsection{Pyroelectricity\buch{650}}
When the temperature of $BaTiO_3$ increases by $\delta T$, the crystal expands
and the relative distances of ions change. This results in a change of 
polarization by $\delta P$. The magnitude of this effect is the 
\emph{pyroelectric coefficient}
\begin{equation}
    p = \frac{\delta P}{\delta P}
\end{equation}

This can be related to a change of the electric field $\delta E$
\begin{equation}
    \delta P = \varepsilon_0 (\varepsilon_r - 1) \delta E
\end{equation}

Some pyroelectric and ferroelectric crystals are depicted in appendix~\ref{app:pyroelectric}.

\subsection{Piezoresistivity\buch{432}}
When a mechanical stress is applied to a sample, it is found that the
resistivity of the sample changes by an amount that depends on the stress.
This \emph{piezoresistivity} is utilized in various sensor applications.

The change in resistivity in semiconductors may be due to a change in the
concentration of carriers or due to a change in the drift mobilities.
In an $n$-type $Si$, the change in the electron mobility $\mu_e$ with
mechanical strain $\varepsilon_m$ is of the order of
$\frac{d\mu_e}{d\varepsilon_m} = \SI{e5}{\farad\centi\meter\squared\per\volt\second}$

The fractional change of the resistivity $\rho$ along the current flow direction is
\begin{equation}
    \frac{\delta\rho}{\rho} = \pi_L \sigma_L + \pi_T \sigma_T
\end{equation}
Where $\sigma_L$ is stress parallel to the current $J_L$ and $\sigma_T$
is stress perpendicular to $J_L$.

The full description of piezoresistivity involves the tensor $\pi_ij$.
In cubic crystals, the three principal coefficients $\sigma_{11}$, $\sigma_{12}$
and $\sigma_{44}$ are sufficient.
$\rho$ and $\Delta\rho$ are tensors of rank 4, $\sigma$ and $\tau$ are the
normal stress and the shear stress.

\begin{equation}
    \Delta\rho = 
    \begin{pmatrix}
        \delta\rho_1\\ \delta\rho_2\\ \delta\rho_3\\ \delta\rho_{12}\\ \delta\rho_{13}\\ \delta\rho_{23}
    \end{pmatrix}
    = \rho_0
    \begin{pmatrix}
        \pi_{11} & \pi_{12} & \pi_{12} & 0 & 0 & 0 \\
        \pi_{12} & \pi_{11} & \pi_{12} & 0 & 0 & 0 \\
        \pi_{12} & \pi_{12} & \pi_{11} & 0 & 0 & 0 \\
        0 & 0 & 0 & \pi_{44} & 0 & 0 \\
        0 & 0 & 0 & 0 & \pi_{44} & 0 \\
        0 & 0 & 0 & 0 & 0 & \pi_{44} \\
    \end{pmatrix}
    \cdot
    \begin{pmatrix}
        \sigma_1 \\ \sigma_2 \\ \sigma_3 \\ \tau_{12} \\ \tau_{13} \\ \tau_{23}
    \end{pmatrix}
\end{equation}

The piezoresistive coefficients in the principal directions are calculated by
\begin{align}
    \pi_L &= \frac{\pi_{11} + \pi_{12} + \pi_{44}}{2} \\
    \pi_T &= \frac{pi_{11} + \pi_{12} - \pi_{44}}{2}
\end{align}

Hooke's law for \emph{isotropic} material is
$\sigma_m = E \varepsilon_m = E \frac{\delta L}{L}$,
where $E$ is the elastic modulus (Young's modulus' 
and $\varepsilon_m$ is the strain.

The piezoresistivity can now be expressed by
\begin{equation}
    \frac{\delta\rho}{\rho} = \pi\sigma_m = \pi E \varepsilon_m = \pi E \frac{\delta L}{L} = K \frac{\delta L}{L}
\end{equation}
where the coefficient $K$ can be found in diagrams.
%TODO: add diagram?