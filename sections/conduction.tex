% !TeX spellcheck = en_US
\section{Electric conduction}

\subsection{Drude model of electrical conduction in solids\buch{114}}

Ohm's law for the electric current density $J$ is
\begin{equation}
    J = \frac{\Delta q}{A \cdot \Delta t}\sigma E = e n v_d
\end{equation}
with the electron density $n = \frac{N}{V}$ the drift velocity
\begin{equation}
    v_d = \frac{e \tau}{m_e} E = \mu_d E \quad \approx \SI{1e-4}{\meter\per\second}
\end{equation}
and drift mobility being
\begin{equation}
    \mu_d = \frac{e \tau}{m_e} = \frac{e l}{m_e u}
\end{equation}

The mean speed $u$ of the electrons in a solid is $u\approx\SI{e6}{\meter\per\second}$.

Conductivity
\begin{equation}
    \frac{1}{\rho} = \sigma = e n \mu_d = \frac{e^2 n \tau}{m_e} = \frac{e^2 n l}{m_e u}
\end{equation}
With the relaxation time
\begin{equation}
    \tau = \frac{1}{\pi a^2 u N_s}
\end{equation}
The mean free path length is
\begin{equation}
    l = u \tau
\end{equation}

It follows that
\begin{equation}
    \sigma = e n \mu_d = \frac{e^2 n l}{m_e u}
\end{equation}

\subsection{Matthiessen's rule\buch{126}}
\begin{equation}
    \rho = \rho_0 \left[ 1 + \alpha_0 \left( T - T_0 \right) \right]
\end{equation}

Resistivity and thermal coefficients for various metals are shown in appendix \ref{app:resistivity}

Frequently, the resistivity can be \emph{empirically} represented by
\begin{equation}
    \rho = \rho_0 \left[ \frac{T}{T_0} \right]^n
\end{equation}

\section{Lorentz Force\buch{145}}

The Lorentz force is
\begin{align}
    \vec{F_L} &= q \vec{v} \times \vec{B} \\
    F_L &= q v B \sin(\theta)
\end{align}
where $q$ is the charge ov the moving particle, $v$ is its velocity,
$B$ is the magnetic field and $\theta$ is the angle between the directions
of $v$ and $B$.

\begin{figure}[ht!]
    \centering
    \begin{tikzpicture}[>=latex]
\begin{scope}[shift={(-3,0)}]
    \node[draw=HSRBlue,fill=HSRBlue80,circle,inner sep=2pt] (nucleus) at (0,0) {\tiny\textcolor{HSRWhite}{$\mathbf{+}$}};
    \node at (0,0.5) {$q=+e$};
    
    \pgfmathparse{1/sqrt(2)};
    \draw[thick,->] (nucleus) -- (1,0) node[anchor=west]{$\vec{v}$};
    \draw[thick,->] (nucleus) -- (-\pgfmathresult,-\pgfmathresult) node[anchor=north east] {$\vec{B}$};
    \draw[thick,->] (nucleus) -- (0,-1.5) node[anchor=north]{$F=q \vec{v} \times \vec{B}$};
\end{scope}
\begin{scope}[shift={(+3,0)}]
    \node[draw=HSRBlue,fill=HSRBlue80,circle,inner sep=2pt] (nucleus) at (0,0) {\tiny\textcolor{HSRWhite}{$\mathbf{+}$}};
    \node at (0,0.5) {$q=-e$};
    
    \pgfmathparse{1/sqrt(2)};
    \draw[thick,->] (nucleus) -- (-1,0) node[anchor=east]{$\vec{v}$};
    \draw[thick,->] (nucleus) -- (-\pgfmathresult,-\pgfmathresult) node[anchor=north east] {$\vec{B}$};
    \draw[thick,->] (nucleus) -- (0,-1.5) node[anchor=north]{$F=q \vec{v} \times \vec{B}$};
\end{scope}
\end{tikzpicture}
    \caption{Lorentz force on a moving charge in a magnetic field}
\end{figure}

\subsection{Hall effect\buch{146}}

The Hall coefficient  is defined as
\begin{equation}
    R_H = \frac{E_z}{J_x B_y} = -\frac{1}{en}
\end{equation}
The Hall voltage $V_H$ and the current $I_x$ are related by the geometry of the metal plate to the current density $J_x$ and the Hall field $E_H$, where $w_z$ is the width and $d_y$ the thickness of the plate.
\begin{align}
    J_x &= \frac{I_x}{w_z d_y} \\
    E_H &= \frac{V_H}{w_z}
\end{align}

Hall coefficients of various metals are depicted in appendix~\ref{app:Hall}.

\paragraph{Semiconductors}~\\
If $n$ and $p$ are the concentrations of electrons and holes in a
semiconductor crystal and have the drift velocities $\mu_e$ and $\mu_h$,
the overall conductivity is given by
\begin{equation}
    \sigma = e p \mu_h + e n \mu_e
\end{equation}

The hall coefficient depends on both carriers and is then
\begin{equation}
    R_H = \frac{p \mu_h^2 - n \mu_e^2}{e\left(p \mu_h + n \mu_e \right)^2}
\end{equation}


\section{Thermal conduction\buch{149}}

Fourier's law of thermal conduction states that
\begin{equation}
	\frac{dQ}{dt} = -A \kappa \frac{\partial T}{\partial x}
\end{equation}

where $\frac{\partial T}{\partial x}$ is the temperature gradient $\nabla T$.
This is similar to Ohm's law:
\begin{equation*}
	\frac{I_x}{A} = J_x = - \sigma \frac{\partial V}{\partial x} = \sigma E_x
\end{equation*}
In metals, $\kappa$ and $\sigma$ originate from free electrons and are related by the Wiedemann~Franz~Lorenz law
\begin{equation}
	\frac{\kappa}{\sigma T} = C_{\mathrm{WFL}} = \text{const.}
\end{equation}
where $C_{\mathrm{WFL}}$ is called the Lorenz number or the Wiedemann~Franz~Lorenz coefficient.
\begin{equation}
	C_{\mathrm{WFL}} = \frac{\pi^2 k_B^2}{3 e^2} = \SI{2.44e-8}{\watt\ohm\per\kelvin\squared}
\end{equation}

The thermal conductivity $\kappa$ for various materials is depicted in appendix~\ref{app:thermalcond}