\section{Electric conduction}

Ohm's law for the electric current density $J$ is
\begin{equation}
    J = \sigma E = e n v_d
\end{equation}
with the drift velocity
\begin{equation}
    v_d = \frac{e \tau}{m_e} E = \mu_d E \quad \approx \SI{1e-4}{\meter\per\second}
\end{equation}
and drift mobility being
\begin{equation}
    \mu_d = \frac{e \tau}{m_e} = \frac{e l}{m_e u}
\end{equation}
Conductivity
\begin{equation}
    \frac{1}{\rho} = \sigma = e n \mu_d = \frac{e^2 n \tau}{m_e} = \frac{e^2 n l}{m_e u}
\end{equation}
With the relaxation time
\begin{equation}
    \tau = \frac{1}{\pi a^2 u N_s}
\end{equation}
The mean free path length is
\begin{equation}
    l = u \tau
\end{equation}

Matthiessen's rule
\begin{equation}
    \rho = \rho_0 \left[ 1 + \alpha_0 \left( T - T_0 \right) \right]
\end{equation}


\section{Thermal conduction}

Fourier's law of thermal conduction states that
\begin{equation}
	\frac{dQ}{dt} = -A \kappa \frac{\partial T}{\partial x}
\end{equation}

where $\frac{\partial T}{\partial x}$ is the temperature gradient $\nabla T$.
This is similar to Ohm's law:
\begin{equation*}
	\frac{I_x}{A} = J_x = - \sigma \frac{\partial V}{\partial x} = \sigma E_x
\end{equation*}
In metals, $\kappa$ and $\sigma$ originate from free electrons and are related by the Wiedemann~Franz~Lorenz law
\begin{equation}
	\frac{\kappa}{\sigma T} = C_{WFL} = \text{const.}
\end{equation}
where $C_{WFL}$ is called the Lorenz number or the Wiedemann~Franz~Lorenz coefficient.
\begin{equation}
	C_{WFL} = \frac{\pi^2 k_B^2}{3 e^2} = \SI{2.44e-8}{\watt\ohm\per\kelvin\squared}
\end{equation}
