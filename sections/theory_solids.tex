\section{Theory of solids}

\subsection{Hydrogen molecule}
Suppose two hydrogen atoms approaching each other
%TODO add image
Each electron in the H atom has a radial 1s wave function $\varPsi \propto e^{-\frac{r}{a_0}}$.
The wave functions overlap as the two atoms approach.
They can overlap either \emph{in phase} or \emph{out of phase}, which leads to the wave functions
\begin{align}
    \varPsi_{\sigma} &= \varPsi_{1s}(r_A) + \varPsi_{1s}(r_B) && \text{bonding} \\
    \varPsi_{\sigma *} &= \varPsi_{1s}(r_A) - \varPsi_{1s}(r_B) && \text{antibonding}
\end{align}
%TODO add image of both possibilities

\begin{table}[htbp]
    \centering
    \begin{tabularx}{0.8\linewidth}{lXXX}
    $\varPsi_{\sigma}$ & No node, symmetric, bonding & High electron density between nuclei & Lower electrostat. energy \\
    $\varPsi_{\sigma *}$ & Node, antisymmetric, antibonding & Low electron density between nuclei & Higher electrostat. energy \\ 
    \end{tabularx}
\end{table}

%TODO add image and text of energy level splitting

Energy level splitting

\subsection{Formation of energy bands}
For a large number of atoms $N$, the separation between energy levels becomes very small, which leads to the formation of energy bands.
%TODO add omre

\subsection{Fermi energy}
A $T=0$ all energy levels up to the \emph{Fermy energy} $E_F$ are full. 
%TODO add picture
The work function $\Phi$ is the energy required to remove an electron from the metal. 
Valence electrons in a metal form an electron gas, i.e. they are free to move around. 
Each electron has a \emph{wavevector} $k$ and a momentum $p = \hbar k$.

The kinetic energy is then
\begin{equation}
    E = \frac{p^2}{2m} = \frac{\hbar^2 k^2}{2m}
\end{equation}

%TODO: image of sp3 orbitals

\subsection{Energy bands in silicon}
%TODO: draw energy bands
In Silicium $N$ $sp^3$ bands overlap to form a valence band (VB) and a conduction band (CB).  
These are separated by the energy gap $E_g$.

\subsection{Effective mass}
%TODO draw
In vacuum, the acceleration of an electron under effect of a force $F_{ext}$ is $a_{vac} = \frac{F_{ext}}{m_e}$. 
In a crystal, internal forces $F_{int}$ exist, which leads to
\begin{equation}
    a_{cryst} = \frac{F_{ext}+F_{int}}{m_e} = \frac{F_{ext}}{m_e^*}
\end{equation}
where $m_e^*$ is the effective mass.

\begin{table}[htbp]
    \centering
    \begin{tabular}{lccccccccccc}
    \toprule
    Metal & Ag & Au & Bi & Cu & K & Li & Na & Ni & Pt & Zn \\
    $\frac{m_e^*}{m_e}$ & 0.99 & 1.10 & 0.047 & 1.01 & 1.12 & 1.28 & 1.2 & 28 & 13 & 0.85 \\
    \bottomrule
    \end{tabular}
    \caption{Effective mass $m_e^*$ of electrons in some metals}
\end{table}

\subsection{Density of states}
The density of states $g(E)$ is the number of states per energy volume. 
$g(E) \propto \sqrt{E}$ at the bottom of the band and $g(E) \propto \sqrt{E_{top}-E}$ at the top of the band.
%TODO: insert image

The distribution of particles (here: electrons) when there are many more available states than there are particles is described by the \emph{Boltzmann energy distribution}.
\begin{equation}
    f(E) \:\propto\: e^{-\frac{E}{kT}}
\end{equation} 

The statistics of electrons in a solid is described by the \emph{Fermi-Dirac distribution}
%TODO insert image
\begin{equation}
    f(E) \:\propto\: \frac{1}{1 + e^{\frac{E-E_F}{kT}}}
\end{equation}
where $f(E)$ is the probability of finding an electron in a state with energy $E$.

%TODO: draw stuff from slide 19--> n_E

The number of valence electrons per unit volume in a metal is
\begin{equation}
    n = \int\limits_{0}^{\text{Top of band}} n_E dE = \int\limits_{0}^{\infty} \frac{\sqrt{E}}{1 + e^{\frac{E-E_F}{kT}}} dE
\end{equation}
which at $T=0K$ leads to
\begin{equation}
    n = \int\limits_{0}^{E_{F0}} \frac{\sqrt{E}}{1+0} dE = E_{F0}^{\frac{3}{2}}
\end{equation}
thus $E_{F0} \propto n^{\frac{2}{3}}$.
A mathematical approximation for $T > 0K$ yields
\begin{equation}
    E_F(t) = E_{F0} \left( 1 - \frac{\pi^2}{12} \left( \frac{kT}{E_{F0}} \right)^2 \right)
\end{equation}

\subsection{Metal-Metal contact potential}
%TODO: add image
When 2 metals are brought together, a \emph{contact potential} $\Phi$ builds up.
An equilibrium is reached when the Fermi levels reach the same value for both metals.

\subsubsection{Thermoelectric effect (Seebeck effect)}
A temperature difference between 2 points in a (semi-) conductor results in a voltage difference.
%TODO add image

The Seebeck coefficient $S$ is calculated from the temperature difference $\Delta T$ and the voltage difference $\Delta V$.
\begin{equation}
    S = \frac{dV}{dT}
\end{equation}

The Seebeck coefficient is dependent on the temperature $T_0$, i.e. $S = \left.\frac{dV}{dT}\right|_{T0}$.
It can be calculated by the Mott and Jones equation
\begin{equation}
    S(T_0) \approx -\frac{\pi^2 k^2 T}{3 e E_{F0}} x
\end{equation}
where $x$ is a numerical constant.

The voltage difference between 2 points can be calculated by
\begin{equation}
    \Delta V = \int\limits_{T_1}^{T_2} S(T) dT
\end{equation}

A table of Seebeck coefficients $S$, Fermi energies $E_{F0}$ and $x$ is listed in table \ref{tab:app_seebeck}.


