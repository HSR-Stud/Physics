% !TeX spellcheck = en_US
\section{Theory of solids}

\subsection{Hydrogen molecule\buch{286}}
Suppose two hydrogen atoms approaching each other
Each electron in the H atom has a radial 1s wave function $\varPsi \propto e^{-\frac{r}{a_0}}$.
The wave functions overlap as the two atoms approach.
They can overlap either \emph{in phase} or \emph{out of phase}, which leads to the wave functions
\begin{align}
    \varPsi_{\sigma} &= \varPsi_{1s}(r_A) + \varPsi_{1s}(r_B) && \text{bonding} \\
    \varPsi_{\sigma *} &= \varPsi_{1s}(r_A) - \varPsi_{1s}(r_B) && \text{antibonding}
\end{align}

\begin{figure}[ht!]
    \centering
    \begin{tikzpicture}[>=latex]
    \begin{scope}[shift={(-8,0)}]
        % Coordinate system
        \draw[->] (0,0) -- (5,0);
        \draw[dashed] (1,0) -- (1,2) node[anchor=south] {$r_A$};
        \draw[dashed] (4,0) -- (4,2) node[anchor=south] {$r_B$};
        
        % Functions 
        \draw[thick,HSRBlue,out=270,in=180] (1,1.6) to (2,0);
        \draw[thick,HSRBlue,out=270,in=0] (1,1.6) to (0,0);
        \draw[thick,HSRBlue,out=270,in=180] (4,1.6) to (5,0);
        \draw[thick,HSRBlue,out=270,in=0] (4,1.6) to (3,0);
        
        % Nodes
        \node at (0,1) {$\varPsi_{1s}(r_A)$};
        \node at (5,1) {$\varPsi_{1s}(r_B)$};
        \node[anchor=north] at (1,0) {A};
        \node[anchor=north] at (4,0) {B};
        \draw[<->] (1,-0.1) to (4,-0.1);
        \node[anchor=north] at (2.5,-0.1) {$R=\infty$};
    \end{scope}
    \begin{scope}
        % Coordinate system
        \draw[->] (0,0) -- (3,0);
        \draw[dashed] (1,0) -- (1,2) node[anchor=south] {$r_A$};
        \draw[dashed] (2,0) -- (2,2) node[anchor=south] {$r_B$};
        
        % Functions 
        \draw[thick,HSRBlue,out=270,in=0] (1,1.6) to (0,0);
        \draw[thick,HSRBlue,out=270,in=180] (1,1.6) to (1.5,0.8);
        \draw[thick,HSRBlue,out=270,in=0] (2,1.6) to (1.5,0.8);
        \draw[thick,HSRBlue,out=270,in=180] (2,1.6) to (3,0);

        % Nodes
        \node[anchor=north] at (1.5,-0.5) {$\varPsi_{\sigma}=\varPsi_{1s}(r_A)+\varPsi_{1s}(r_B)$};
        \node[anchor=north] at (1.5,-1) {in phase};
        \node[anchor=north] at (1,0) {A};
        \node[anchor=north] at (2,0) {B};
        \draw[<->] (1,-0.1) to (2,-0.1);
        \node[anchor=north] at (1.5,-0.1) {a};      
    \end{scope}
    \begin{scope}[shift={(+4,0)}]
        % Coordinate system
        \draw[->] (0,0) -- (3,0);
        \draw[dashed] (1,0) -- (1,2) node[anchor=south] {$r_A$};
        \draw[dashed] (2,0) -- (2,2) node[anchor=south] {$r_B$};
        
        % Functions 
        \draw[thick,HSRBlue,out=270,in=0] (1,2) to (0,1);
        \draw[thick,HSRBlue,out=270,in=180] (1,2) to (1.5,1);
        \draw[thick,HSRBlue,out=90,in=0] (2,0) to (1.5,1);
        \draw[thick,HSRBlue,out=90,in=180] (2,0) to (3,1);

        % Nodes
        \node[anchor=north] at (1.5,-0.5) {$\varPsi_{\sigma}=\varPsi_{1s}(r_A)-\varPsi_{1s}(r_B)$};
        \node[anchor=north] at (1.5,-1) {out of phase};
        \node[anchor=north] at (1,0) {A};
        \node[anchor=north] at (2,0) {B};
        \draw[<->] (1,-0.1) to (2,-0.1);
        \node[anchor=north] at (1.5,-0.1) {a};      
    \end{scope}
\end{tikzpicture}
    \caption{Molecular wavefunctions of two hydrogen atoms}
\end{figure}

\begin{table}[ht!]
    \centering
    \begin{tabularx}{0.8\linewidth}{lXXX}
    \toprule
    $\varPsi_{\sigma}$ & No node, symmetric, bonding & High electron density between nuclei & Lower electrostat. energy \\
    $\varPsi_{\sigma *}$ & Node, antisymmetric, antibonding & Low electron density between nuclei & Higher electrostat. energy \\ 
    \bottomrule
    \end{tabularx}
\end{table}

\paragraph{Energy level splitting} 
is due to interaction (overlap) between atomic orbitals.
The atomic energy level $E_{1s}$ splits into 2 energy levels $E_{\sigma}$ and $E_{\sigma^*}$.

\begin{figure}[ht!]
    \centering
    \begin{tikzpicture}[>=latex,scale=0.8]
    % H-atoms
    \draw[ultra thick] (0,0) node[anchor=east]{$E_{1s}$} -- (1,0);
    \draw[ultra thick] (4,0) -- (5,0)node[anchor=west]{$E_{1s}$};
    \draw[->] (0.5,-0.2) -- (0.5,0.2); 
    \draw[->] (4.5,-0.2) -- (4.5,0.2);
    \node at (0.5,-1.5) {\ce{H}-atom};
    \node at (4.5,-1.5) {\ce{H}-atom};
    
    % H2
    \draw[ultra thick,HSRBlue] (2,1) -- (3,1) node[anchor=west] {$E_{\sigma^*}$};
    \draw[ultra thick,HSRBlue] (2,-1) -- (3,-1) node[anchor=west] {$E_{\sigma}$};
    \draw[->] (2.4,-1.2) -- (2.4,-0.8); 
    \draw[->] (2.6,-1.2) -- (2.6,-0.8);
    \node at (2.5,-1.5) {\ce{H2}};
    
    % Connect
    \draw[dashed]	(1,0) -- (2,1)
                    (1,0) -- (2,-1)
                    (3,1) -- (4,0)
                    (3,-1) -- (4,0);
    
\end{tikzpicture}
    \caption{Energy level splitting in a $H_2$ molecule}
\end{figure}

\subsection{Formation of energy bands\buch{291}}
For a large number of atoms $N$, the separation between energy levels becomes very small, which leads to the formation of energy bands.
As each energy level can take 2 electrons, the energy band is half full.

\paragraph{Metals}
In metals, the energy bands overlap to give a single band of energies.

\subsection{Fermi energy\buch{295}}
A $T=0$ all energy levels up to the \emph{Fermy energy} $E_F$ are full. 

\begin{figure}[ht!]
    \centering
    \begin{tikzpicture}[>=latex]
    %Coordinate system
    \draw (0,0) -- (2,0);
    \draw[->] (0,0) -- (0,4) node[anchor=south] {Electron energy};
    \node[anchor=east] at (-0.5,0) {$E_B$};
    \node[anchor=east] at (-0.5,2) {$E_{F0}$};
    \node[anchor=east] at (-0.5,3) {Vacuum Level};
    
    % levels
    \draw (0,2) -- (2,2) -- (2,0);
    \draw (0,3) -- (2,3) -- (2,2);
    
    % electrons
    \foreach \x in {0.1,0.3,...,1.9} {
        \foreach \y in {0.1,0.3,...,1.9} {
            \node[fill=HSRBlue,circle,inner sep=1.8pt]  at (\x,\y) {};
        }
    }
    
    \draw[|<->|] (2.5,0) -- (2.5,2) node[midway,anchor=west] {$E_{F0}$};
    \draw[|<->|] (2.5,2) -- (2.5,3) node[midway,anchor=west] {$\varPhi$};
    
    % outside metal
    \draw[|->] (1,3.05) -- (1,3.8);
    \node[anchor=west] at (1,3.5) {Electron outside the metal};
\end{tikzpicture}
    \caption{Typical electron energy band for a metal}
\end{figure}

The work function $\Phi$ is the energy required to remove an electron from the metal. 
The Fermi energy and work function of selected metals is depicted in appendix~\ref{app:fermienergy}.

Valence electrons in a metal form an electron gas, i.e. they are free to move around. 
Each electron has a \emph{wavevector} $k$ and a momentum $p = \hbar k$.

The kinetic energy is then
\begin{equation}
    E = \frac{p^2}{2m} = \frac{\hbar^2 k^2}{2m}
\end{equation}
%TODO: probably add parabola

If no electric field is applied, the same number of electrons is moving to the right
and to the left. 
In the presence of an electric field, the momentum is flipped and there is a
net electric current.

\subsection{Energy bands in silicon\buch{299}}
In Silicium $N$ $sp^3$ bands overlap to form a valence band (VB) and a conduction band (CB).  
These are separated by the energy gap $E_g$.
At $T=0$ the valence band is full and the conduction band is empty.

\begin{figure}[ht!]
    \centering
    \begin{tikzpicture}[>=latex]
    % Isolated Si atom
    \draw[ultra thick,HSRBlue] (0,1) node[anchor=east] {$3p$} -- (1,1);
    \draw[ultra thick,HSRBlue] (0,0) node[anchor=east] {$3s$} -- (1,0);
    \draw[->] (0.3,0.8) -- (0.3,1.2);
    \draw[->] (0.8,0.8) -- (0.8,1.2);
    \draw[->] (0.3,-0.2) -- (0.3,0.2);
    \draw[<-] (0.4,-0.2) -- (0.4,0.2);
    \node[anchor=north,text width=1.5cm] at (0.5,-1.5) {Isolated Si atom};
    
    % connect
    \draw[dashed] (1,1) -- (1.5,0.5) -- (1,0);
    
    % sp3 hybridization
    \draw[ultra thick,HSRBlue] (1.5,0.5) -- (2.5,0.5);
    \foreach \x in {1.7,1.9,...,2.3}{
        \draw[->] (\x,0.3) -- (\x,0.8);
    }
    \node[anchor=north,text width=1.5cm] at (2.2,-1.5) {$sp^3$ hybridization};
    
    % connect
    \draw[dashed] (3.5,-0.5) -- (2.5,0.5) -- (3.5,1.5);
    
    %Interaction of sp3 orbitals
    \draw[ultra thick,HSRBlue] (3.5,-0.5) -- (4.5,-0.5);
    \draw[ultra thick,HSRBlue] (3.5,1.5) -- (4.5,1.5);
    \node[anchor=north] at (3.5,-0.5) {bonding};
    \node[anchor=south] at (3.5,1.5) {antibonding};
    \draw[->] (3.8,-0.7) -- (3.8,-0.3);
    \draw[<-] (3.9,-0.7) -- (3.9,-0.3);
    \node[anchor=north,text width=1.5cm] at (4.0,-1.5) {Interaction of 2 $sp^3$ orbitals};

    % connect
    \draw[dashed] (5,2.0) -- (4.5,1.5) -- (5,1.0);
    \draw[dashed] (5,0) -- (4.5,-0.5) -- (5,-1.0);
    
    % bands
    \draw (5,0) rectangle (7,-1);
    \draw[fill=black!20] (5,2) rectangle (7,1);
    \foreach \x in {5.1,5.3,...,6.9} {
        \foreach \y in {-0.1,-0.3,...,-0.9} {
            \node[fill=HSRBlue,circle,inner sep=1.8pt]  at (\x,\y) {};
        }
    }
    \node[anchor=west] at (7,-0.5) {Valence band};
    \node[anchor=west] at (7,1.5) {Conduction band};
    \draw[<->] (6.8,0) -- (6.8,1) node[midway,anchor=west]{Energy gap $E_g$};
    \node[anchor=north,text width=1.5cm] at (6.2,-1.5) {Band formation};

\end{tikzpicture}
    \caption{Energy bands in silicon}
\end{figure}

By an excitation process (e.g. thermal excitation), an electron in the CB and
a hole in the VB can be generated.

\subsection{Electron effective mass\buch{303}}
In vacuum, the acceleration of an electron under effect of a force $F_{ext}$ is $a_{vac} = \frac{F_{ext}}{m_e}$. 
In a crystal, internal forces $F_{int}$ exist, which leads to
\begin{equation}
    a_{cryst} = \frac{F_{ext}+F_{int}}{m_e} = \frac{F_{ext}}{m_e^*}
\end{equation}
where $m_e^*$ is the effective mass.
A table of the effective mass for some metals is in appendix~\ref{app:effectivemass}.

\subsection{Density of states\buch{305}}
The density of states $g(E)$ is the number of states per energy volume. 
$g(E) \propto \sqrt{E}$ at the bottom of the band and $g(E) \propto \sqrt{E_{top}-E}$ at the top of the band.

\paragraph{Distribution of electrons\buch{312}}
The distribution of particles (here: electrons) when there are many more available
 states than there are particles is described by the 
 \emph{Boltzmann energy distribution}.
\begin{equation}
    f(E) \:\propto\: e^{-\frac{E}{kT}}
\end{equation} 

\paragraph{Statistics of electrons in solids\buch{315}}
The statistics of electrons in a solid is described by the \emph{Fermi-Dirac distribution}
\begin{equation}
    f(E) \:\propto\: \frac{1}{1 + e^{\frac{E-E_F}{kT}}}
\end{equation}
where $f(E)$ is the probability of finding an electron in a state with energy $E$.

\begin{figure}[ht!]
    \centering
    %TODO: maybe fill stuff
    \begin{tikzpicture}

\begin{scope}[shift={(-10,0)}]
    \begin{axis}[
        width=6cm, height=6cm,
        domain=-10:10, samples=51,
        xtick={0,0.5,1},ytick={1},
        yticklabels={$E_{F}$},
        legend pos=outer north east,
        xlabel={$f(E)$}, ylabel={$E$},
    ]
        \addplot[HSRBlue] ({1/(1+exp((x-1)/0.001))},x);
        \addplot[] ({1/(1+exp((x-1)/1))},x);
        \addplot[dashed] ({1/(1+exp((x-1)/2))},x);

        \legend{$T=0$,$T_1>0$,$T_2>T_1$}
        
    \end{axis}
\end{scope}

\begin{scope}[shift={(0,+1)}]
    \draw (-0.5,2) -- (0,0) -- (1,0) -- (1.5,2);
    \draw[HSRBlue] (-0.375,1.5) -- (1.375,1.5);
    \node[anchor=north] at (0.5,0) {$T=0$};
    
\end{scope}

\begin{scope}[shift={(+4,+1)}]
    \draw (-0.5,2) -- (0,0) -- (1,0) -- (1.5,2);
    \draw[HSRBlue]	(-0.375,1.5)	sin
                    (-0.1563,1.625)	cos
                    (0.0625,1.5)	sin
                    (0.2813,1.375)	cos
                    (0.5000,1.5)	sin
                    (0.7188,1.625)	cos
                    (0.9375,1.5)	sin
                    (1.1563,1.375)	cos
                    (1.375,1.5);
    \node[anchor=north] at (0.5,0) {$T>0$};
\end{scope}

\end{tikzpicture}
    \caption{Fermi-Dirac distribution}
\end{figure}

%TODO: draw stuff from slide 19--> n_E

The number of valence electrons per unit volume in a metal is
\begin{equation}
    n = \int\limits_{0}^{\text{Top of band}} n_E dE = \int\limits_{0}^{\infty} \frac{\sqrt{E}}{1 + e^{\frac{E-E_F}{kT}}} dE
\end{equation}
which at $T=0K$ leads to
\begin{equation}
    n = \int\limits_{0}^{E_{F0}} \frac{\sqrt{E}}{1+0} dE = E_{F0}^{\frac{3}{2}}
\end{equation}
thus $E_{F0} \propto n^{\frac{2}{3}}$.
A mathematical approximation for $T > 0K$ yields
\begin{equation}
    E_F(t) = E_{F0} \left( 1 - \frac{\pi^2}{12} \left( \frac{kT}{E_{F0}} \right)^2 \right)
\end{equation}

\paragraph{Electrons in a semiconductor}
%TODO: draw stuff from slide 21

\subsection{Metal-Metal contact potential\buch{320}}
When 2 metals are brought together, a \emph{contact potential} $\Phi$ builds up.
An equilibrium is reached when the Fermi levels reach the same value for both metals.

\begin{figure}[ht!]
    \centering
    \begin{tikzpicture}

\begin{scope}[shift={(-4,0)}]

    %Pt
    \node[anchor=south] at (1,2) {Pt};
    \draw[draw=black,fill=HSRBlue] (0,0) rectangle (2,0.5);
    \draw[draw=black] (0,0.5) rectangle (2,2);
    \draw[|<->|] (-0.5,2) -- (-0.5,0.5);
    \node[anchor=south,rotate=90] at (-0.6,1) {$\Phi$=\SI{5.36}{\electronvolt}};
    
    %Mo
    \node[anchor=south] at (4,2) {Mo};
    \draw[draw=black,fill=HSRBlue] (3,0) rectangle (5,1.5);
    \draw[draw=black] (3,1.5) rectangle (5,2);
    \draw[|<->|] (5.5,2) -- (5.5,1.5);
    \node[anchor=north,rotate=90] at (5.6,1) {$\Phi$=\SI{4.20}{\electronvolt}}; 
    
    \draw[dashed] (2,0.5) -- (3,1.5);   
\end{scope}
\begin{scope}[shift={(+4,0)}]

    \draw[draw=black,fill=HSRBlue] (0,0) rectangle (2,0.5);
    \draw[draw=black] (0,0.5) rectangle (2,2);
    \draw[draw=black,fill=HSRBlue] (2,-1) rectangle (4,0.5);
    \draw[draw=black] (2,0.5) rectangle (4,1);
    
    \draw[|<->|] (2.5,1.0) -- (2.5,2) node[midway,anchor=west] {$e\Delta V$=\SI{1.16}{\electronvolt}};
    
    \node[white] at (2.2,0.25) {\tiny\textbf{+}};
    \node[white] at (1.8,0.25) {\tiny\textbf{-}};

\end{scope}

\end{tikzpicture}
    \caption{Metal-metal contact potential}
\end{figure}

\subsubsection{Thermoelectric effect (Seebeck effect)\buch{322}}
A temperature difference between 2 points in a (semi-) conductor results in a voltage difference.
%TODO add image

The Seebeck coefficient $S$ is calculated from the temperature difference $\Delta T$ and the voltage difference $\Delta V$.
\begin{equation}
    S = \frac{dV}{dT}
\end{equation}

The Seebeck coefficient is dependent on the temperature $T_0$, i.e. $S = \left.\frac{dV}{dT}\right|_{T0}$.
It can be calculated by the Mott and Jones equation
\begin{equation}
    S(T) \approx -\frac{\pi^2 k^2 T}{3 e E_{F0}} x
\end{equation}
where $x$ is a numerical constant.

The voltage difference between 2 points can be calculated by
\begin{equation}
    \Delta V = \int\limits_{T_1}^{T_2} S(T) dT
\end{equation}

A table of Seebeck coefficients $S$, Fermi energies $E_{F0}$ and $x$ is listed in table \ref{app:seebeck}.


