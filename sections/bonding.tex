\section{Bonding}
Balance between attractive force (Coulomb force) and repulsive force (shell overlap).

\paragraph{Molecular bonding}
%TODO: add graph

\paragraph{Ionic bonding}
%TODO: add images / example

\begin{equation}
	E(r) = -\frac{e^2 M}{\varepsilon_0 4 \pi r} + \frac{B}{r^m}
\end{equation}
where $M$, $B$ and $m$ are constants. 

$M$ is the Mandelung constant, which models Coulomb interactions in the ionic crystal.
It depends on the crystal type and is $M=1.748$ for a \ce{NaCl} type (FCC).

\paragraph{Covalent bonding} 
Atoms share electrons to fill their valence bands.

\paragraph{Metallic bonding}
Atoms are held together by electron gas.

\paragraph{Secondary (Van der Waals) bonding}
Weak force between polar molecules (dipoles, e.g. \ce{H2O}). 

\paragraph{Comparison} ~\\
%TODO include table 1.2 (kasap, page 21)
\begin{tabularx}{\linewidth}{llllllX}
	& Typical Solids & Bond Energy  & Melt. Temp. & Elastic Modulus & Density & Typical Properties \\ \toprule
	Ionic & NaCl & 3.2 & 801 & 40 & 2.17 & \\
	
	\bottomrule
\end{tabularx}

\subsection{Crystal structures}